%% LyX 2.4.0~RC3 created this file.  For more info, see https://www.lyx.org/.
%% Do not edit unless you really know what you are doing.
\documentclass[spanish]{article}
\usepackage[T1]{fontenc}
\usepackage{textcomp}
\usepackage[latin9]{inputenc}
\usepackage{geometry}
\geometry{verbose,tmargin=2cm,bmargin=2cm,lmargin=2cm,rmargin=2cm}
\usepackage{color}
\usepackage{babel}
\addto\shorthandsspanish{\spanishdeactivate{~<>}}
\deactivatequoting

\PassOptionsToPackage{normalem}{ulem}
\usepackage{ulem}
\usepackage[pdfusetitle,
 bookmarks=true,bookmarksnumbered=false,bookmarksopen=false,
 breaklinks=true,pdfborder={0 0 1},backref=false,colorlinks=true]
 {hyperref}

\makeatletter
%%%%%%%%%%%%%%%%%%%%%%%%%%%%%% User specified LaTeX commands.
\usepackage{charter}

\makeatother

\begin{document}
\title{Analysis of Genomic Selection Variying Markers}
\maketitle

\section{Material and Methods}

\subsection{GP using marker densities}

{\scriptsize\textcolor{red}{\sout{To identify the optimal number of markers for predicting GEBVs for each LCH component of the color traits, genomic prediction was performed using varying SNP densities. Markers were selected based on GWAS results and ranked by importance using our custom GSCORE scoring function (see Section 2.3).}}}{\scriptsize\par}

\section{Results}

\subsection{GP using marker densities}

Genomic prediction was performed using marker subsets representing 5\% to 100\% of the total 4,641 SNPs. Predictive ability, assessed by the Pearson correlation between observed phenotypes and genomic estimated breeding values (GEBVs), varied by trait and marker density (see Figure S4).

In general, predictive ability increased rapidly with marker density up to \textbackslash\textasciitilde 40--50\% of the total set, after which gains plateaued---indicating diminishing returns. This pattern held across most color traits and LCH components (Hue, Chroma, Lightness), suggesting that a moderate number of well-distributed, informative SNPs can achieve predictive performance comparable to that of the full marker set.

Stem color and primary flower color traits consistently showed higher predictive abilities, with correlations nearing 0.75 at full marker density. In contrast, secondary traits---such as secondary tuber flesh or sprout color---showed lower predictive values, even with more markers. This variability reflects differences in genetic architecture, with some traits likely controlled by major loci, and others influenced by more polygenic and complex effects.

These results support the feasibility of cost-effective genomic selection in tetraploid {*}Andigenum{*} potatoes using an optimized SNP subset. The top 50\% of markers, selected via the GSCORE metric (Section 2.3), effectively captured relevant genetic variation, enabling accurate predictions while reducing computational demand.

\section{ChatGpt Editting:}

To evaluate the efficiency of genomic prediction models across different SNP densities, genomic selection (GS) was performed using subsets of markers representing 5\%, 10\%, 20\%, 30\%, 40\%, 50\%, 60\%, 70\%, 80\%, and 100\% of the total 4,641 markers. Predictive ability, measured as the Pearson correlation between observed phenotypes and genomic estimated breeding values (GEBVs), varied depending on the trait and the number of markers used.

In general, predictive ability increased rapidly with the inclusion of additional SNPs up to approximately 40--50\% of the marker set. Beyond this threshold, the gains in predictive ability plateaued, indicating diminishing returns with additional markers. This trend was consistent across most color traits and LCH components (Hue, Chroma, and Lightness), suggesting that a moderate number of well-distributed and informative SNPs can yield prediction accuracies comparable to those obtained using the complete marker set.

Among the evaluated traits, components of stem color and primary flower color consistently demonstrated higher predictive abilities, with correlations reaching values close to 0.75 when using 100\% of the markers. In contrast, secondary traits, such as the secondary tuber flesh or sprout colors, exhibited lower predictive values even at higher marker densities. These results indicate variability in the genetic architecture of the color traits, where some traits are likely controlled by fewer loci with larger effects, while others may be more polygenic and influenced by complex interactions.

Overall, these findings highlight the potential for cost-effective genomic selection in tetraploid Andigenum potatoes by utilizing an optimized number of markers. The top 50 SNPs, selected via our GSCORE metric (Section 2.3), captured sufficient genetic architecture. This approach allows for improved selection accuracy without the computational burden of using the entire marker set.

\section{DeepSeek Edditing:}

We evaluated the predictive ability of genomic selection (GS) for 27 Hue-Chroma-Lightness (HCL) traits derived from nine color characteristics in tetraploid Andigenum potatoes. Using the BWGS pipeline (Charmet et al., 2020), we implemented 12 GS models spanning parametric (GBLUP, EGBLUP, RR-BLUP, LASSO), semiparametric (RKHS, Random Forest, SVM), and Bayesian approaches (BRR, BL, BA, BB, BC). Five-fold cross-validation repeated five times was employed to assess model performance.

\textbf{Key findings:}
\begin{enumerate}
\item \textbf{Full Marker Set Performance (4,641 SNPs):}
\begin{itemize}
\item Predictive abilities (correlation between observed and predicted values) varied substantially among traits (Figure 5B). Primary flower color (PCFlower) and stem color (StemC) showed the highest predictive abilities (0.86 and 0.75, respectively), while secondary tuber flesh color (SCTuberflesh) had the lowest (0.18).
\item The RKHS model outperformed others for 11/27 HCL components, followed by EGBLUP (9/27) and SVM (5/27). Bayesian models showed particular effectiveness for traits with complex inheritance patterns.
\end{itemize}
\item Optimization of Marker Number:
\begin{itemize}
\item Analysis of progressively reduced marker sets (100, 50, 35, 15, and 5 top-ranked SNPs) revealed that predictive ability plateaued at \textasciitilde 50 markers for most traits (Supplementary Figure S3). Notably:
\item Primary tuber skin color (PCTuberskin) maintained 92\% of full-set predictive ability with just 50 markers
\item Secondary sprout color (SCSprout) required 100 markers to reach maximum performance
\item The top 50 SNPs, selected via our GSCORE metric (Section 2.3), captured sufficient genetic architecture while reducing computational load by 98.9\%.
\end{itemize}
\item Trait Architecture Insights:
\begin{itemize}
\item Traits with highest predictive abilities (PCFlower, StemC) also exhibited the highest narrow-sense heritabilities (h\texttwosuperior{} > 0.8, Figure 5A), suggesting predominantly additive genetic control.
\item Poorly predicted traits (e.g., SCTuberflesh) showed lower heritability (h\texttwosuperior{} = 0.42) and higher environmental influence, consistent with their weaker GWAS signals.
\end{itemize}
\item Component-Specific Patterns:
\begin{itemize}
\item Hue (H) components were generally more predictable than Chroma (C) or Lightness (L), with mean predictive abilities of 0.61 vs. 0.49 and 0.47, respectively. This aligns with known biochemical pathways where hue variation strongly reflects anthocyanin and carotenoid content.
\end{itemize}
\end{enumerate}
\textbf{Breeding Implications:}

The demonstrated efficacy of GS with reduced marker sets (50-100 SNPs) enables cost-effective implementation in Andigenum potato breeding programs. Our results suggest:
\begin{itemize}
\item Early-generation selection for highly heritable traits (flower/stem color) can achieve accuracy >0.75
\item Secondary color traits may require phenotypic validation despite genomic predictions
\item The RKHS model is recommended as default for HCL trait prediction, with EGBLUP for hue-related components
\end{itemize}

\end{document}
